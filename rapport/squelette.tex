\documentclass{article}

\usepackage{a4wide}
\usepackage[utf8]{inputenc}
\usepackage[T1]{fontenc}
\usepackage[french]{babel}
\usepackage[babel=true]{csquotes} % guillemets français
\usepackage{graphicx}
\graphicspath{{Images/}}
\usepackage{color}
\usepackage{hyperref}
\hypersetup{colorlinks,linkcolor=,urlcolor=blue}

\usepackage{amsmath}
\usepackage{amssymb}


\title{Raport}
\author{\'Said Ismael, M1 Informatique}
\date{\today}

\begin{document}

\maketitle % pour écrire le titre


%% Le résumé:
\begin{abstract}
  Travail réalisé avec le \textit{Cirad}~\cite{coursera}
\end{abstract}


\section{Introduction}
\section{Domaine}
Définition de l'ensemble des termes associés au SGBD \\ 
\textbf{essaie:Définition}
\\\textbf{modalité:Définition}
\section{Modèle entité association}
uml 

\section{modèle relationnelle}
Conversion:\\
Essaie(\underline{codeEssai},nomEssai)\\
Adventice(\underline{nomAdventice}) \\
Parcelle(\underline{codeParcelle},bloc,nomParcelle)\\
EssaiParcelle(\underline{codeParcelle},\underline{codeEssai},\underline{dateDebut})\\
Facteur(\underline{codeFacteur},descriptionFacteur)\\
Modalite(\underline{codeFacteur},\underline{codeParcelle},\underline{dateApplicationModalite})\\




%%% La bibliographie:
\bibliographystyle{plain}
\bibliography{ma_biblio}


\end{document}
