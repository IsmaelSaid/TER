\documentclass{article}

\usepackage{a4wide}
\usepackage[utf8]{inputenc}
\usepackage[T1]{fontenc}
\usepackage[french]{babel}
\usepackage[babel=true]{csquotes} % guillemets français
\usepackage{graphicx}
\graphicspath{{Images/}}
\usepackage{color}
\usepackage{hyperref}
\hypersetup{colorlinks,linkcolor=,urlcolor=blue}

\usepackage{amsmath}
\usepackage{amssymb}


\title{Raport}
\author{Said Ismael, M1 Informatique}
\date{\today}

\begin{document}

\maketitle % pour écrire le titre

% PAGE BLANCHE

\newpage
\thispagestyle{empty}
\mbox{}
\newpage

% FIN PAGE BLANCHE



% TABLE DES MATIERES
\tableofcontents
\newpage

% PAGE BLANCHE

\section{Introduction}
%% Le résumé:
\begin{abstract}
  Travail réalisé avec le \textit{Cirad}~\cite{coursera}
\end{abstract}



\section{Modèle entité association}
Définition de l'ensemble des termes associés au SGBD \\ 
\textbf{essaie:} Définition
\\\textbf{modalité:} Définition \\
Sémantique des relations:
\begin{itemize}
  \item Un essai appartient au minimum à un seul projet et au maximum a n projets 
  \item Un Projet contient au minimum un seul essai et au maximum n essais
  \item Un essai met en jeu au minimum une seule parcelle et au maximum n parcelles
  \item Une parcelle est le sujet de un seul et unique essai
  \item Une adventice peut se retrouver au minimum sur aucune et au maximum sur n parcelles
  \item Une parcelle peut contenir au minimum aucune adventice et au maximim n adventices
  
\end{itemize}


\section{modèle relationnelle}
Conversion:\\
Essaie(\underline{codeEssai},nomEssai)\\
Adventice(\underline{nomAdventice}) \\
Parcelle(\underline{codeParcelle},bloc,nomParcelle)\\
EssaiParcelle(\underline{codeParcelle},\underline{codeEssai},\underline{dateDebut})\\
Facteur(\underline{codeFacteur},descriptionFacteur)\\
Modalite(\underline{codeFacteur},\underline{codeParcelle},\underline{dateApplicationModalite})\\




%%% La bibliographie:
\bibliographystyle{plain}
\bibliography{ma_biblio}


\end{document}
